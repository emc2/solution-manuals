Consider a uniform thin disk that rolls without slipping on a
horizontal plane.  A horizontal force is applied to the center of the
disk and in a direction parallel to the plane of the disk.
\begin{itemize}
\item[(a)] Derive Lagrange's equations and find the generalized force.
\item[(b)] Discuss the motion if the force is not applied parallel to the
  plane of the disk.
\end{itemize}

\subsection*{Solution}
\begin{itemize}
\item[(a)]
  For the first part, define the positive $x$ axis as the direction of
  the disk's motion, $a$ as the disk's radius, $m$ as its mass, and
  $\theta$ as its angular displacement (in radians).

  For any point in the disk at radius $r$ and angle $\theta$ from a line
  through the center parallel to the $x$ axis, we have a rotational
  velocity $\velocity_R(r, \theta) = [\dot{x_R}(r, \theta), \dot{y_R}(r,
    \theta)]$, where
  \begin{align*}
    \dot{x_r}(r, \theta) &= r\dot{\theta}\sin{\theta} \\
    \dot{y_r}(r, \theta) &= r\dot{\theta}\cos{\theta} \\
  \end{align*}
  Now, the position of the center of the disk, $x_c$ is also a
  parameter.  Since the disk is rigid, this is a component of the
  velocities of all points.  Thus, we have
  \[ \velocity(r, \theta) = \velocity_R(r, \theta) + \velocity_c \]
  Since the disk rolls without slipping, it must be the case that the
  point at which the disk touches the ground does not move; therefore,
  we have
  \[ \velocity_R(a, \pi/2) + \velocity_c = 0 \]
  This tells us that
  \[ \velocity_c = [a\dot{\theta}, 0] \]
  and we have
  \[
  \velocity(r, \theta) =
  [
    r\dot{\theta}\sin{\theta} + a\dot{\theta},
    r\dot{\theta}\cos{\theta}
  ]
  \]
  and
  \begin{align*}
    \dotprod{\velocity}{\velocity}
    &= r^2\dot{\theta}^2\sin^2{\theta} +
    ar\dot{\theta}^2\sin{\theta} +
    a^2\dot{\theta}^2 +
    r^2\dot{\theta}^2\cos^2{\theta} \\
    &= r^2\dot{\theta}^2(\cos^2{\theta} + \sin^2{\theta}) +
    ar\dot{\theta}^2\sin{\theta} +
    a^2\dot{\theta}^2 \\
    &= r^2\dot{\theta}^2 +
    ar\dot{\theta}^2\sin{\theta} +
    a^2\dot{\theta}^2 \\
  \end{align*}

  Since the disk has uniform density $\rho = \frac{m}{2\pi a^2}$, we can
  find the total kinetic energy using the integral
  \begin{align*}
    \int \frac{1}{2}\rho \dotprod{\velocity}{\velocity} dA
    &=
    \frac{m}{2\pi a^2}
    \int
    r^2\dot{\theta}^2 +
    ar\dot{\theta}^2\sin{\theta} +
    a^2\dot{\theta}^2
    dA
    \end{align*}
  Now, the measure of area in terms of radial and angular displacement
  is
  \[ dA = rd\theta dr \]
  Therefore, we apply Fubini's theorem to get
  \[
  \frac{m}{2\pi a^2} \int_0^a \int_0^{2\pi}
  r^3\dot{\theta}^2 +
  r^2a\dot{\theta}^2\sin{\theta} +
  ra^2\dot{\theta}^2
  d\theta dr
  \]
  Note that $\dot{\theta}$ is independent of $\theta$ here, so the
  integral by $\theta$ is trivial, giving
  \begin{align*}
    \frac{m}{2\pi a^2} \int_0^a \int_0^{2\pi}
    r^3\dot{\theta}^2 +
    r^2a\dot{\theta}^2\sin{\theta} +
    ra^2\dot{\theta}^2
    d\theta dr
    &=
    \frac{m}{a^2} \int_0^a dr
    r^3\dot{\theta}^2 +
    ra^2\dot{\theta}^2 \\
    &=
    \frac{m}{a^2}
    \left(\frac{1}{4} a^4\dot{\theta}^2 +
    \frac{1}{2} a^4\dot{\theta}^2\right) \\
    &=
    \frac{1}{4}ma^2\dot{\theta}^2 + \frac{1}{2} ma^2\dot{\theta}^2 \\
    &=
    \frac{3}{4}ma^2\dot{\theta}^2
  \end{align*}

  Now, find the key derivatives
  \begin{align*}
    \partderiv{T}{\theta}
    &=
    \partderivop{\theta}\frac{3}{4}ma^2\dot{\theta}^2 \\
    &=
    0
  \end{align*}
  \begin{align*}
    \partderiv{T}{\dot{\theta}}
    &=
    \partderivop{\dot{\theta}}\frac{3}{4}ma^2\dot{\theta}^2 \\
    &=
    \frac{3}{2}ma^2\dot{\theta}
  \end{align*}
  \begin{align*}
    \derivop{t}\partderiv{T}{\dot{\theta}}
    &=
    \derivop{t}\frac{3}{2}ma^2\dot{\theta} \\
    &=
    \frac{3}{2}ma^2\ddot{\theta}
  \end{align*}

  To find the generalized force, take the force $\force$ applied to the
  center of the disk and apply the formula
  \[Q_{\theta} = \dotprod{\force}{\partderiv{x_c}{\theta}}\]
  Where $x_c$ is displacement of the centerUse the fact that
  \[\partderiv{\radius_i}{q_j} = \partderiv{\velocity_i}{\dot{q_j}}\]
  to get
  \[\dotprod{F}{\partderiv{\dot{x}}{\dot{\theta}}}\]
  From the relationship between velocity of the center to angular
  velocity, we have
  \[\dot{x_c} = a\dot{\theta}\]
  therefore
  \begin{align*}
    \partderiv{\dot{x_c}}{\dot{\theta}}
    &=
    \partderivop{\dot{\theta}}a\dot{\theta} \\
    &=
    a
  \end{align*}
  Therefore, we get
  \[ Q_{\theta} = a\force \]
  (This is essentially the torque resulting from the force)

  Putting all this together gives us the Lagrange equation
  \[ \frac{3}{2}ma^2\ddot{\theta} = a\force \]
  or
  \[ \frac{3}{2}ma\ddot{\theta} = \force \]

\item[(b)]
  If the force is not parallel to the $x$ axis (which we have defined as
  the axis of motion), then some component of it (which may be 0) acts
  in the $x$ direction, some component acts perpendicular to the disk at
  its exact center, and some component acts on the vertical axis.  Since
  the disk cannot slip and the perpendicular force acts on the exact
  center, there is no way to turn the disk.  Since there is no friction,
  and we assume the disk cannot leave contact with the surface, the
  force on the vertical axis similarly has no effect.  This leaves only
  the $x$ component of the generalized force, which acts as before,
  giving us
  \[ \frac{3}{2}ma\ddot{\theta} = \force_x \]

\end{itemize}
