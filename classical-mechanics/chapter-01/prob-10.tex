Let $q_1,\ldots,q_n$ be a set of independent generalized coordinates
for a system of $n$ degrees of freedom, with a Lagrangian $L(q,
\dot{q}, t)$.  Suppose we transform to another set of independent
coordinates $s_1,\ldots,s_n$ by means of transformation equations
\[q_i = q_i(s_1, \ldots, s_n), i\in[1,n]\]
Show that if the Lagrangian function is expressed as a function of
$s_j,\dot{s_j}$, and $t$ through the equations of transformation, then
L satisfies Lagrange's equations with respect to the $s$ coordinates:
\[ \derivop{t}\left(\partderiv{L}{\dot{s_j}}\right) - \partderiv{L}{s_j} = 0 \]
In other words, the form of Lagrange's equations is invariant under a
point transformation.

\subsection*{Solution}

First, examine the key derivatives.  Start with $\dot{q_i}$ in terms
of $s_1,\ldots,s_n$
\begin{align*}
  \dot{q_j} &= \deriv{q_j}{t} \\
  &= \sum_k\partderiv{q_j}{s_k}\dot{s_k} + \partderiv{q_j}{t}
\end{align*}
First, note that
\[\partderiv{q_j}{\dot{s_k}} = 0\]
since no $q_j$ may depend on any $\dot{s_k}$.

Now examine $\partderiv{\dot{q_j}}{s_k}$
\[ \partderivop{s_k}\deriv{q_j}{t} \]
We can apply the identity
\[ \partderivop{x}\deriv{f}{t} = \derivop{t}\partderiv{f}{x} \]
from Devrivation~\ref{nielsen} to obtain
\[ \partderiv{\dot{s_k}}{q_j} = \derivop{t}\partderiv{s_k}{q_j} \]

Finally, for $\partderiv{\dot{q_j}}{\dot{s_k}}$
\[ \partderivop{\dot{s_k}}\deriv{q_j}{t} \]
apply the identity
\[
\derivop{t}\partderiv{f}{\dot{x}}
=
\partderivop{\dot{x}}\deriv{f}{t} - \partderiv{f}{\dot{x}}
\]
from Devrivation~\ref{nielsen} to obtain
\[ \derivop{t}\partderiv{q_j}{\dot{s_k}} + \partderiv{q_j}{s_k} \]
But we know that
\[\partderiv{q_j}{\dot{s_k}} = 0\]
therefore, we obtain
\[ \partderiv{\dot{q_j}}{\dot{s_k}} = \partderiv{q_j}{s_k} \]
Now use these to examine $\partderiv{L}{s_j}$ and
$\derivop{t}\partderiv{L}{\dot{s_j}}$.  For $\partderiv{L}{s_j}$, we
have
\[
\sum_k\partderiv{L}{\dot{q_k}}\partderiv{\dot{q_k}}{s_j}
+\sum_k\partderiv{L}{q_k}\partderiv{q_k}{s_j}\]
apply the identity $\partderiv{\dot{q_j}}{\dot{s_k}} =
\partderiv{q_j}{s_k}$ to get
\[
\sum_k\partderiv{L}{\dot{q_k}}\left(\derivop{t}\partderiv{q_k}{s_j}\right)
+\sum_k\partderiv{L}{q_k}\partderiv{q_k}{s_j}
\]

For $\derivop{t}\partderiv{L}{\dot{s_j}}$, first examine
$\partderiv{L}{\dot{s_j}}$
\[
\sum_k\partderiv{L}{\dot{q_k}}\partderiv{\dot{q_k}}{\dot{s_j}}
+\sum_k\partderiv{L}{q_k}\partderiv{q_k}{\dot{s_j}}
\]
we can apply the identities $\partderiv{q_j}{\dot{s_k}} = 0$ to
eliminate the second term, and then $\partderiv{\dot{q_j}}{\dot{s_k}}
= \partderiv{q_j}{s_k}$ to obtain
\[\sum_k\partderiv{L}{\dot{q_k}}\partderiv{q_k}{s_j}\]
Now examine the total derivative of the summation element
\[\derivop{t}\left(\partderiv{L}{\dot{q_k}}\partderiv{q_k}{s_j}\right)\]
\[
\left(\derivop{t}\partderiv{L}{\dot{q_k}}\right)\partderiv{q_k}{s_j}
+\partderiv{L}{\dot{q_k}}\left(\derivop{t}\partderiv{q_k}{s_j}\right)
\]

Now use these to examine the Lagrange equation in terms of $s_j$
\begin{align*}
  \derivop{t}\partderiv{L}{\dot{s_j}} - \partderiv{L}{s_j}
  &=
  \left[
    \sum_k\left(\derivop{t}\partderiv{L}{\dot{q_k}}\right)\partderiv{q_k}{s_j} +
    \sum_k\partderiv{L}{\dot{q_k}}\left(\derivop{t}\partderiv{q_k}{s_j}\right)
  \right]
  -\left[
    \sum_k\partderiv{L}{\dot{q_k}}\left(\derivop{t}\partderiv{q_k}{s_j}\right) +
    \sum_k\partderiv{L}{q_k}\partderiv{q_k}{s_j}
  \right] \\
  &=
  \sum_k\left(\derivop{t}\partderiv{L}{\dot{q_k}}\right)\partderiv{q_k}{s_j}
  -\sum_k\partderiv{L}{q_k}\partderiv{q_k}{s_j} \\
  &=
  \sum_k
  \left[
    \left(\derivop{t}\partderiv{L}{\dot{q_k}}\right)\partderiv{q_k}{s_j}
    -\partderiv{L}{q_k}\partderiv{q_k}{s_j}
  \right] \\
  &=
  \sum_k\left(\derivop{t}\partderiv{L}{\dot{q_k}}
  -\partderiv{L}{q_k}\right)\partderiv{q_k}{s_j}
\end{align*}
We can apply the Lagrange equation on $q_k$ to the inner term.  This
shows the entire right-hand side to be 0, thus giving us
\[\derivop{t}\partderiv{L}{\dot{s_j}} - \partderiv{L}{s_j} = 0\]
