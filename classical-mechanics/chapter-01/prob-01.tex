Show that for a single particle with constant mass the equation of
motion implies the following differential equation for the kinetic energy:
\[\deriv{T}{t} = \dotprod{\force}{\velocity}\]
while if the mass varies with time the corresponding equation is
\[\deriv{(mT)}{t} = \dotprod{\force}{\momentum}\]

\subsection*{Solution}

\subsubsection*{Solution for Constant Mass}

Start with
\[T = \int\dotprod{\force}{d\disp}\]
Change the integration variable to $t$:
\begin{align*}
  T
  &=
  \int\dotprod{\force}{\deriv{\disp}{t}}dt \\
  &=
  \int\dotprod{\force}{\velocity}dt \\
\end{align*}
Then differentiate both sides with respect to time to obtain
\[
\deriv{T}{t}
=
\derivop{t}\int\dotprod{\force}{\velocity}dt
\]
The Fundamental Theorem of Calculus then gives us
\[
\deriv{T}{t}
=
\dotprod{\force}{\velocity}
\]
as desired.

\subsubsection*{Solution for Varying Mass}

Start with
\[
\int\dotprod{\force}{\momentum}dt
=
\sum_i\int F_i p_i dt
\]
Substitute by $\force = \deriv{\momentum}{t}$:
\[
\int\dotprod{\force}{\momentum}dt
=
\sum_i\int p_i \deriv{p_i}{t}dt
\]
Change integration varibles to $p_i$:
\[
\int\dotprod{\force}{\momentum}dt
=
\sum_i\int p_i dp_i
\]
This gives us:
\[
\int\dotprod{\force}{\momentum}dt
=
\sum_i\frac{p_i^2}{2}
\]
Substitute $\momentum = m\velocity$:
\[
\int\dotprod{\force}{\momentum}dt
=
\sum_i\frac{mv_i^2}{2}
\]
Substitute $T = \sum_i\frac{m^2v_i^2}{2}$:
\[
mT
=
\int\dotprod{\force}{\momentum}dt
\]
Now, differentiate both sides with respect to $t$ to get
\[
\deriv{(mT)}{t}
=
\derivop{t}
\int\dotprod{\force}{\momentum}dt
\]
The Fundamental Theorem of Calculus gives us:
\[
\deriv{(mT)}{t}
=
\dotprod{\force}{\momentum}
\]
