The electromagnetic field is invariant under a gauge transformation of
the scalar and vector potential given by
\begin{align*}
  \vecpotential &\rightarrow \vecpotential + \grad{\psi(\radius, t)} \\
  \phi &\rightarrow \phi + \partderiv{\psi}{t}
\end{align*}
where $\psi$ is arbitrary but differentiable.  What effect does this
gauge transformation have on the Lagrangian of a particle moving in
the electromagnetic field?  Is the motion affected?

Note: errata applied, removing a $\frac{1}{c}$ factor.

\subsection*{Solution}
As per equation 1.63, the Lagrangian is
\[ L = \frac{1}{2}mv^2 - q\phi + q\dotprod{\vecpotential}{\velocity} \]
Apply the transformation to get
\begin{align*}
  L
  &=
  \frac{1}{2}mv^2 - q\phi + q\dotprod{\vecpotential}{\velocity} \\
  &=
  \frac{1}{2}mv^2 -
  q\left(\phi + \partderiv{\psi}{t}\right) +
  q\dotprod{\left(\vecpotential + \grad{\psi(\radius, t)}\right)}{\velocity} \\
  &=
  \frac{1}{2}mv^2 -
  q\phi +
  q\dotprod{\vecpotential}{\velocity} +
  q\partderiv{\psi}{t} +
  q\dotprod{\grad{\psi(\radius, t)}}{\velocity}
\end{align*}
Expand the terms of
\[ \dotprod{\grad{\psi(\radius, t)}}{\velocity} \]
to get
\begin{align*}
  \dotprod{\grad{\psi(\radius, t)}}{\velocity}
  &=
  \dotprod{\left[ \partderiv{\psi(\radius, t)}{x}, \partderiv{\psi(\radius, t)}{y}, \partderiv{\psi(\radius, t)}{z} \right]}{[\dot{x}, \dot{y}, \dot{z}]} \\
  &=
  \partderiv{\psi(\radius, t)}{x}\dot{x}
  +\partderiv{\psi(\radius, t)}{y}\dot{y}
  +\partderiv{\psi(\radius, t)}{z}\dot{z}
\end{align*}
Substituting this into the Lagrangian gives us
\begin{align*}
  L
  &=
  \frac{1}{2}mv^2 -
  q\phi +
  q\dotprod{\vecpotential}{\velocity} +
  q\partderiv{\psi}{t} +
  q\dotprod{\grad{\psi(\radius, t)}}{\velocity} \\
  &=
  \frac{1}{2}mv^2 -
  q\phi +
  q\dotprod{\vecpotential}{\velocity} +
  q\left(
  \partderiv{\psi(\radius, t)}{x}\dot{x} +
  \partderiv{\psi(\radius, t)}{y}\dot{y} +
  \partderiv{\psi(\radius, t)}{z}\dot{z} +
  \partderiv{\psi}{t}
  \right) \\
  &=
  \frac{1}{2}mv^2 -
  q\phi +
  q\dotprod{\vecpotential}{\velocity} +
  q\deriv{\psi(\radius, t)}{t}
\end{align*}
Stated in terms of the original Lagrangian, this is
\[ L + q\deriv{\psi(\radius, t)}{t} \]
which is a transform of the kind demonstrated in
Derivation~\ref{lagrangeinv} to also satisfy Lagrange's equations.
Note that this transform does not affect the velocity; therefore,
motion is unaffected.
