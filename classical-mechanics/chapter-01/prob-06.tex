A particle moves in the xy plane under the constraint that its velocity
vector is always directed towards a point on the $x$ axis whose
x-coordinate is some given function of time $f(t)$.  Show that for
$f(t)$ differentiable, but otherwise abirtrary, the constraint is
nonholonomic.

\subsection*{Solution}

Let $\radius = [x, y]$ be the position of the particle and $\velocity
= [\dot{x}, \dot{y}]$ be its velocity. We have that the velocity must
always be parallel to $[x - f(t), y]$, which gives us
\[ \dotprod{\velocity}{[-y, x - f(t)]} = 0 \]
Thus
\begin{align*}
\dot{y}(x - f(t)) - \dot{x}y &= 0 \\
(x - f(t))dy - ydx &= 0
\end{align*}
This gives us a constrant system as in Derivation~\ref{rollingdisk}, where
\begin{align*}
  g_x(x, y, f(t)) &= y \\
  g_y(x, y, f(t)) &= x - f(t) \\
  g_{f(t)}(x, y, f(t)) &= 0
\end{align*}
We need to find $h(x, y, f(t))$ such that for all $i \neq j \in \{x,
y, \theta, \phi\}$,
\[\partderivop{x_j}hg_i = \partderivop{x_i}hg_j\]
Examining all possibilities gives three equations
\begin{align*}
  \partderivop{x}h(x - f(t)) &= \partderivop{y}hy \\
  \partderivop{f(t)}h(x - f(t)) &= 0 \\
  \partderivop{f(t)}hy &= 0 \\
\end{align*}
The only non-constant derivative is
\[\partderivop{x}h(x - f(t)) = \partderivop{y}hy\]
Rewrite this as
\begin{align*}
  \partderivop{x}h(x - f(t))
  &=
  \partderivop{y}hy \\
  h\partderiv{x}{x} - h\partderivop{x}f(t) + (x - f(t))\partderiv{h}{x}
  &=
  h\partderiv{y}{y} + y\partderiv{h}{y} \\
  (x - f(t))\partderiv{h}{x}
  &=
  y\partderiv{h}{y}
\end{align*}
To satisfy this, it must be true that $\partderiv{h}{y}$ contains an
additive component of $f(t)$.  However, if we examine the third
equation, we get
\begin{align*}
  \partderivop{f(t)}hy &= 0 \\
  \partderiv{h}{f(t)} + \partderiv{y}{f(t)} &= 0 \\
  \partderiv{h}{f(t)} &= 0
\end{align*}
which shows that $h$ \emph{cannot} contain such a term.  Therefore,
there exists no satisfactory integrating factor, and thus the
constraints are not holonomic.
