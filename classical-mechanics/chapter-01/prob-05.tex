Two wheels of radius $a$ are mounted on the ends of a common axle of
length $b$ such that the wheels rotate independently.  The whole
combination rolls without slipping on a plane.  Show that there are
two nonholonomic equations of constraint,
\begin{align*}
\cos\theta dx + \sin\theta dy &= 0 \\
\sin\theta dx - \cos\theta dy &= \frac{1}{2}a(d\phi + d\phi')
\end{align*}
(where $\theta$, $\phi$, and $\phi'$ have meanings similar to those in
the problem of a single vertical disk, and $(x, y)$ are the
coordinates of a point on the axle midway between the two wheels) and
one holonomic constraint,
\[\theta = C - \frac{a}{b}(\phi - \phi')\]
where $C$ is a constant.

\subsection*{Solution}

Let $\theta$ be the angle of the axle to the $x$ axis, $\phi$, $\phi'$
be the angles of rotation of the two wheels, $x$ and $y$ be the center
of the axle.

The centerpoint of the axle cannot move along the axle.  Therefore, we
have
\[\dotprod{[\cos\theta, \sin\theta]}{\velocity} = 0\]
which is
\[\cos\theta\deriv{x}{t} + \sin\theta\deriv{y}{t} = 0\]
This gives us the constraint
\[\cos\theta dx + \sin\theta dy = 0\]
Therefore, we have that the velocity vector is orthogonal to the axle:
\[\dotprod{[\sin\theta, -\cos\theta]}{\velocity} = |\velocity|\]
\[\sin\theta\deriv{x}{t} - \cos\theta\deriv{y}{t} = |\velocity|\]

It therefore remains to determine the magnitude of the velocity
vector.  If we assume that both wheels have the same mass, and the
axle has uniform density, then we can derive the center of motion:
\[\momentum = \momentum_1 + \momentum_2\]
Decompose momentums:
\[2m\velocity = m\velocity_1 + m\velocity_2\]
Then we get:
\[\velocity = \frac{1}{2}(\velocity_1 + \velocity_2)\]
Both $\velocity_1$ and $\velocity_2$ have the same direction, so
\[|\velocity| = \frac{1}{2}(|\velocity_1| + |\velocity_2|)\]
The magnitudes of the velocities of the wheels are $a\deriv{\phi}{t}$
and $a\deriv{\phi'}{t}$, so we have:
\begin{align*}
  |\velocity|
  &=
  \frac{1}{2}\left(a\deriv{\phi}{t} + a\deriv{\phi'}{t}\right) \\
  &=
  \frac{a}{2}\left(\deriv{\phi}{t} + \deriv{\phi'}{t}\right)
\end{align*}
So we have
\[\sin\theta\deriv{x}{t} - \cos\theta\deriv{y}{t} =
\frac{1}{2}a\left(\deriv{\phi}{t} + \deriv{\phi'}{t}\right)\]
This gives us the constraint
\[\sin\theta dx - \cos\theta dy = \frac{1}{2}a(d\phi + d\phi')\]

Now, given the total angles of rotation of the wheels $\phi$ and
$\phi'$ (in radians), the total distance traveled is $a\phi$ and
$a\phi'$ respectively.  Unless $\phi = \phi'$, both wheels travel a
circular path with a total angular displacement of $\theta'$.

Let $b'$ be the radius of the wheel whose displacement is $\phi'$, and
$b + b'$ be the radius of the wheel whose displacement is $\phi$, with
either radius being negative if the corresponding wheel's displacement
is negative.  Then we have:
\begin{align*}
  b'\theta' &= a\phi' \\
  (b' + b)\theta' &= a\phi
\end{align*}
Rewrite the second as
\[b'\theta = a\phi - b\theta' \]
Combine the two to get
\begin{align*}
  a\phi'
  &=
  a\phi - b\theta' \\
  b\theta'
  &=
  a\phi - a\phi' \\
  \theta'
  &=
  \frac{a}{b}(\phi - \phi')
\end{align*}
Now, set the initial angle to $C$; since the wheel rolls clockwise,
the angular displacement is subtracted from this to give the final
angle, giving us
\[\theta = C - \frac{a}{b}(\phi - \phi')\]
