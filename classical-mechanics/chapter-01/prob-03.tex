Suppose a system of two particles is known to obey the equations of motion,
\[M\secondderiv{\physvec{R}}{t} =
\sum_i\force^{(e)}_i \equiv
\force^{(e)}\]
\[\deriv{\angmomentum}{t} = \torque^{(e)}\]
From the equations of motion of the individual particles show that the
internal forces between particles satisfy both the weak and strong
laws of action and reaction.

\subsection*{Solution}

To begin, for force, we have
\begin{align*}
  M\secondderiv{\physvec{R}}{t}
  &=
  \force^{(e)} \\
  &=
  \force^{(e)}_1 + \force^{(e)}_2
\end{align*}
and
\[
\force_1 = \force^{(e)}_1 + \force_{21},\qquad
\force_2 = \force^{(e)}_2 + \force_{12},\qquad
m_i\secondderiv{\radius_i}{t} = \force_i
\]
and for torque, we have
\begin{align*}
  \deriv{\angmomentum}{t}
  &=
  \torque^{(e)} \\
  &=
  \crossprod{\physvec{R}}{\force^{(e)}} \\
  &=
  \torque^{(e)}_1 + \torque^{(e)}_2 \\
  &=
  \crossprod{\radius_1}{\force^{(e)}_1} + \crossprod{\radius_1}{\force^{(e)}_1}
\end{align*}

\subsubsection*{Weak Law of Action}

Start with
\[\force_1 + \force_2\]
Apply $\force_i = m_i\secondderiv{\radius_i}{t}$ to get
\begin{align*}
  \force_1 + \force_2
  &=
  m_1\secondderiv{\radius_1}{t} + m_2\secondderiv{\radius_2}{t} \\
  &=
  \secondderivop{t}(m_1\radius_1 + m_2\radius_2)
\end{align*}
Use $\physvec{R} = \frac{m_1r1 + m_2\radius_2}{M}$ to get
\[M\secondderiv{\physvec{R}}{t}\]
Now we have
\[M\secondderiv{\physvec{R}}{t} = \force_1 + \force_2\]
Rewrite using $\force_1 = \force^{(e)}_1 + \force_{21}$ and $\force_2 =
\force^{(e)}_2 + \force_{12}$ to get
\begin{align*}
  M\secondderiv{\physvec{R}}{t}
  &=
  \force^{(e)}_1 + \force_{21} + \force^{(e)}_2 + \force_{12} \\
  &=
  \force^{(e)} + \force_{12} + \force_{21}
\end{align*}
Apply $M\secondderiv{\physvec{R}}{t} = \force^{(e)}$ to get
\[\force_{12} + \force_{21} = 0\]
This gives us the weak law of action:
\[\force_{12} = -\force_{21}\]

\subsubsection*{Strong Law of Action}

Given
\[\force_1 = \force^{(e)}_1 + \force_{12}\]
define
\[\torque_1 =
\crossprod{\radius_1}{\force_1} =
\crossprod{\radius_1}{\force^{(e)}_1} + \crossprod{\radius_1}{\force_{12}}\]
Then apply $\torque^{(e)}_i = \crossprod{\radius_i}{\force^{(e)}_i}$ to get
\[
\torque_1 = \torque^{(e)}_1 + \crossprod{\radius_1}{\force_{12}},\qquad\qquad
\torque_2 = \torque^{(e)}_2 + \crossprod{\radius_2}{\force_{21}}
\]

Now start with
\[\torque_1 + \torque_2\]
Apply $\torque_i = \crossprod{\radius_i}{\force_i}$ to get
\begin{align*}
  \torque_1 + \torque_2
  &=
  \crossprod{\radius_i}{\force_i} + \crossprod{\radius_i}{\force_i} \\
  &=
  \derivop{t}\crossprod{\radius_1}{\momentum_1}
  +\derivop{t}\crossprod{\radius_2}{\momentum_2} \\
  &=
  \derivop{t}
  \left(
  \crossprod{\radius_1}{\momentum_1}
  +\crossprod{\radius_2}{\momentum_2}
  \right)
\end{align*}
then apply $\angmomentum = \crossprod{\radius_1}{\momentum_1} +
\crossprod{\radius_2}{\momentum_2}$ to get
\[\torque_1 + \torque_2 = \deriv{\angmomentum}{t}\]
Apply the derivations $\torque_1 = \torque^{(e)}_1 +
\crossprod{\radius_1}{\force_{12}}$ and $\torque_2 = \torque^{(e)}_2 +
\crossprod{\radius_2}{\force_{21}}$
\[
\torque^{(e)}_1
+\crossprod{\radius_1}{\force_{12}}
+\torque^{(e)}_2
+\crossprod{\radius_2}{\force_{21}}
=
\deriv{\angmomentum}{t}
\]
Apply $\torque^{(e)} = \torque^{(e)}_1 + \torque^{(e)}_2$ and $\deriv{\angmomentum}{t} = \torque^{(e)}$ to get
\[
\crossprod{\radius_1}{\force_{12}} + \crossprod{\radius_2}{\force_{21}}
=
0
\]
Apply $\force_{12} = -\force_{21}$ and $\radius_2 = \radius_1 +
\radius_{12}$ to get
\begin{align*}
  \crossprod{\radius_1}{\force_{12}}
  -\crossprod{(\radius_1 + \radius_{12})}{\force_{12}}
  &=
  0 \\
  \crossprod{\radius_{12}}{\force_{12}}
\end{align*}
Which by $\force_{12} = -\force_{21}$ and $\radius_{12} =
-\radius_{21}$ also gives us
\[\crossprod{\radius_{21}}{\force_{21}} = 0\]
