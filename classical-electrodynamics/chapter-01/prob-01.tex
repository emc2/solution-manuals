Use Gauss' theorem and
\[
\oint \physvec{E}\cdot d\physvec{l} = 0
\]
to prove the following:
\begin{itemize}
\item[(a)] Any excess charge placed on a conductor must lie entirely
  on its surface.  (A conductor by definition contains charges
  capable of moving freely under the action of applied electric
  fields).
\item[(b)] A closed, hollow conductor shields its interior from
  fields due to charges outside, but does not shield its exterior
  from the fields due to charges placed inside it.
\item [(c)] The electric field at the surface of a conductor is
  normal to the surface and has a magnitude of
  $\chargedensity/\permittivity$, where $\chargedensity$ is the
  charge density per unit area on the surface.
\end{itemize}

\subsection*{Solution}

\begin{itemize}
\item[(a)] In any system in equilibrium,for each component of the system,
  the measure of work done by some spontaneous force acting on the
  component $dW_s$ must equal the sum of the measure of work done by
  the forces of constraint $dW_c$ and the change in energy $dE$ of the
  component, giving
  \[
  dW_s = dW_c + dE
  \]
  If we consider on the component of the spontaneous force that acts
  against the forces of constraint, then we have
  \[
  dW_s' = dW_c
  \]
  Since a conductor allows free motion of charges, the only
  constraining forces from the $\elecfield$ field.  The work done
  for some displacement from $x_1$ to $x_2$ is given by
  \[
  \int_{x_1}^{x_2} dW_c
  =
  \int_{x_1}^{x_2} \elecpotential dl
  =
  \elecpotential(x_2) - \elecpotential(x_1)
  \]
  Thus, we have
  \[
  \int_{x_1}^{x_2} dW_s = \elecpotential(x_2) - \elecpotential(x_1)
  \]
  Finally, by d'Alembert's principle of virtual work, $dW_s' = 0$ for any
  system in equilibrium, so we have
  \[
  \elecpotential(x_2) - \elecpotential(x_1) = 0
  \]
  Since $x_1,x_2$ are arbitrary points inside the conductor, we
  conclude that the potential within the conductor is some constant
  $\phi_c$.  From the equation $\elecfield = \nabla\elecpotential$, we
  then have
  \[
  \elecfield = \nabla\elecpotential = \nabla\phi_c = 0
  \]
  everywhere within the interior of the conductor.

  Now, consider any surface $S$ confined to the interior of the
  conductor.  By Gauss' law, we have that
  \[
  \frac{Q}{\permittivity} = \oint_S \elecfield \cdot dA
  \]
  But since $\elecfield = 0$ everywhere in the interior of the
  conductor, we have
  \[
  Q = 0
  \]
  Since $S$ is any arbitrary surface contained within the conductor,
  we therefore conclude that the charge within the whole interior must
  be zero, and therefore that all charges are confined to the boundary.


\item[(b)]

  First, examine the case where the inner space (the space contained
  by the conductor) contains no charges.  Then for any surface $S$
  contained within the interior of the inner space, we have
  \[
  \oint_S \elecfield \cdot dA = \frac{Q}{\permittivity} = 0
  \]
  Thus, the total electric flux at any point in the inner space is
  zero.  Now, consider a surface $S'$ in the interior of the
  \emph{conductor} (which means it contains the inner space entirely).
  Since the charges in a conductor are confined to the boundary, this
  means that the total charge contained by $S'$ is equal to the charge
  $Q_I$ of the inner boundary.  But since $\elecfield = 0$ everywhere
  within a conductor's interior, and $S'$ is confined to the interior,
  we have that
  \[
  \frac{Q_c}{\permittivity} = \oint_{S'} \elecfield \cdot dA = 0
  \]
  Thus, there are no charges on the interior boundary, and thus
  nowhere within $S'$.  Since there are no charges within $S'$, the
  electric field must be uniform throughout.  But since the portion of
  $S'$ that lies within the conductor is known to have $\elecfield =
  0$, then the electric field must be zero everywhere.

  Now, place some charge $Q$ within the inner space, and consider some
  surface $S$ such that the conductor's outer boundary is wholly
  contained within $S$.  Then by Gauss' law, we have
  \[
  \oint_S \elecfield \cdot dA = \frac{Q}{\permittivity}
  \]
  which implies $\elecfield$ must be non-zero outside the conductor.

\item[(c)]

  Let $S$ be any surface that encloses an area of boundary having
  total charge $Q$ and area $A$.  Then by Gauss' theorem, we have
  \[
  \oint_S \elecfield \cdot dA = \frac{Q}{\permittivity}
  \]
  We can replace $Q$ with a surface integral over the portion of the
  boundary of the conductor $S_c$ contained by $S$, giving
  \[
  \oint_S \elecfield \cdot dA
  =
  \frac{1}{\permittivity}\int_{S_c}\chargedensity dA
  \]
  From this, we obtain that
  \[
  \elecfield\cdot\normvec{n} = \frac{\chargedensity}{\permittivity}
  \]
  where $\normvec{n}$.  Now, consider some contour $C$ that passes through
  the interior of the conductor, then through the exterior, and back.
  Split this into two parts
  \[
  \int_{C_I} \elecfield\cdot dl
  \]
  which passes through the interior of the conductor, and
  \[
  \int_{C_O} \elecfield\cdot dl
  \]
  which lies outside of the conductor.  Since $\elecfield = 0$ inside
  the conductor, we have
  \[
  \int_{C_I} \elecfield\cdot dl = 0
  \]
  Since
  \[
  \oint_C \elecfield\cdot dl = 0
  \]
  and
  \[
  \oint_C \elecfield\cdot dl
  =
  \int_{C_I} \elecfield\cdot dl +
  \int_{C_O} \elecfield\cdot dl
  \]
  then it must be true that
  \[
  \int_{C_O} \elecfield\cdot dl = 0
  \]
  For the component of $\elecfield$ in the direction of $\normvec{n}$,
  rewrite this integral as
  \[
  \int_{x_1}^{x_2} \elecfield\cdot\normvec{n}dl
  =
  \elecpotential(x_2) - \elecpotential(x_1)
  \]
  But since the component of $\elecfield$ in the direction of
  $\normvec{n}$ is uniform in the region of the contour, we have
  \[
  \elecpotential(x_2) = \elecpotential(x_1)
  \]
  and thus
  \[
  \int_{C_O} \elecfield\cdot\normvec{n}dl
  =
  0
  \]
  Now consider the component of $\elecfield$ orthogonal to $\normvec{n}$,
  then for some contour, we will have that
  \[
  \int_{C_O} \elecfield\cdot dl
  \neq
  0
  \]
  Thus, $\elecfield$ must have no component orthogonal to $\normvec{n}$.

\end{itemize}
